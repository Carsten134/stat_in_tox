% Options for packages loaded elsewhere
\PassOptionsToPackage{unicode}{hyperref}
\PassOptionsToPackage{hyphens}{url}
%
\documentclass[
]{article}
\usepackage{amsmath,amssymb}
\usepackage{iftex}
\ifPDFTeX
  \usepackage[T1]{fontenc}
  \usepackage[utf8]{inputenc}
  \usepackage{textcomp} % provide euro and other symbols
\else % if luatex or xetex
  \usepackage{unicode-math} % this also loads fontspec
  \defaultfontfeatures{Scale=MatchLowercase}
  \defaultfontfeatures[\rmfamily]{Ligatures=TeX,Scale=1}
\fi
\usepackage{lmodern}
\ifPDFTeX\else
  % xetex/luatex font selection
\fi
% Use upquote if available, for straight quotes in verbatim environments
\IfFileExists{upquote.sty}{\usepackage{upquote}}{}
\IfFileExists{microtype.sty}{% use microtype if available
  \usepackage[]{microtype}
  \UseMicrotypeSet[protrusion]{basicmath} % disable protrusion for tt fonts
}{}
\makeatletter
\@ifundefined{KOMAClassName}{% if non-KOMA class
  \IfFileExists{parskip.sty}{%
    \usepackage{parskip}
  }{% else
    \setlength{\parindent}{0pt}
    \setlength{\parskip}{6pt plus 2pt minus 1pt}}
}{% if KOMA class
  \KOMAoptions{parskip=half}}
\makeatother
\usepackage{xcolor}
\usepackage[margin=1in]{geometry}
\usepackage{color}
\usepackage{fancyvrb}
\newcommand{\VerbBar}{|}
\newcommand{\VERB}{\Verb[commandchars=\\\{\}]}
\DefineVerbatimEnvironment{Highlighting}{Verbatim}{commandchars=\\\{\}}
% Add ',fontsize=\small' for more characters per line
\usepackage{framed}
\definecolor{shadecolor}{RGB}{248,248,248}
\newenvironment{Shaded}{\begin{snugshade}}{\end{snugshade}}
\newcommand{\AlertTok}[1]{\textcolor[rgb]{0.94,0.16,0.16}{#1}}
\newcommand{\AnnotationTok}[1]{\textcolor[rgb]{0.56,0.35,0.01}{\textbf{\textit{#1}}}}
\newcommand{\AttributeTok}[1]{\textcolor[rgb]{0.13,0.29,0.53}{#1}}
\newcommand{\BaseNTok}[1]{\textcolor[rgb]{0.00,0.00,0.81}{#1}}
\newcommand{\BuiltInTok}[1]{#1}
\newcommand{\CharTok}[1]{\textcolor[rgb]{0.31,0.60,0.02}{#1}}
\newcommand{\CommentTok}[1]{\textcolor[rgb]{0.56,0.35,0.01}{\textit{#1}}}
\newcommand{\CommentVarTok}[1]{\textcolor[rgb]{0.56,0.35,0.01}{\textbf{\textit{#1}}}}
\newcommand{\ConstantTok}[1]{\textcolor[rgb]{0.56,0.35,0.01}{#1}}
\newcommand{\ControlFlowTok}[1]{\textcolor[rgb]{0.13,0.29,0.53}{\textbf{#1}}}
\newcommand{\DataTypeTok}[1]{\textcolor[rgb]{0.13,0.29,0.53}{#1}}
\newcommand{\DecValTok}[1]{\textcolor[rgb]{0.00,0.00,0.81}{#1}}
\newcommand{\DocumentationTok}[1]{\textcolor[rgb]{0.56,0.35,0.01}{\textbf{\textit{#1}}}}
\newcommand{\ErrorTok}[1]{\textcolor[rgb]{0.64,0.00,0.00}{\textbf{#1}}}
\newcommand{\ExtensionTok}[1]{#1}
\newcommand{\FloatTok}[1]{\textcolor[rgb]{0.00,0.00,0.81}{#1}}
\newcommand{\FunctionTok}[1]{\textcolor[rgb]{0.13,0.29,0.53}{\textbf{#1}}}
\newcommand{\ImportTok}[1]{#1}
\newcommand{\InformationTok}[1]{\textcolor[rgb]{0.56,0.35,0.01}{\textbf{\textit{#1}}}}
\newcommand{\KeywordTok}[1]{\textcolor[rgb]{0.13,0.29,0.53}{\textbf{#1}}}
\newcommand{\NormalTok}[1]{#1}
\newcommand{\OperatorTok}[1]{\textcolor[rgb]{0.81,0.36,0.00}{\textbf{#1}}}
\newcommand{\OtherTok}[1]{\textcolor[rgb]{0.56,0.35,0.01}{#1}}
\newcommand{\PreprocessorTok}[1]{\textcolor[rgb]{0.56,0.35,0.01}{\textit{#1}}}
\newcommand{\RegionMarkerTok}[1]{#1}
\newcommand{\SpecialCharTok}[1]{\textcolor[rgb]{0.81,0.36,0.00}{\textbf{#1}}}
\newcommand{\SpecialStringTok}[1]{\textcolor[rgb]{0.31,0.60,0.02}{#1}}
\newcommand{\StringTok}[1]{\textcolor[rgb]{0.31,0.60,0.02}{#1}}
\newcommand{\VariableTok}[1]{\textcolor[rgb]{0.00,0.00,0.00}{#1}}
\newcommand{\VerbatimStringTok}[1]{\textcolor[rgb]{0.31,0.60,0.02}{#1}}
\newcommand{\WarningTok}[1]{\textcolor[rgb]{0.56,0.35,0.01}{\textbf{\textit{#1}}}}
\usepackage{graphicx}
\makeatletter
\def\maxwidth{\ifdim\Gin@nat@width>\linewidth\linewidth\else\Gin@nat@width\fi}
\def\maxheight{\ifdim\Gin@nat@height>\textheight\textheight\else\Gin@nat@height\fi}
\makeatother
% Scale images if necessary, so that they will not overflow the page
% margins by default, and it is still possible to overwrite the defaults
% using explicit options in \includegraphics[width, height, ...]{}
\setkeys{Gin}{width=\maxwidth,height=\maxheight,keepaspectratio}
% Set default figure placement to htbp
\makeatletter
\def\fps@figure{htbp}
\makeatother
\setlength{\emergencystretch}{3em} % prevent overfull lines
\providecommand{\tightlist}{%
  \setlength{\itemsep}{0pt}\setlength{\parskip}{0pt}}
\setcounter{secnumdepth}{-\maxdimen} % remove section numbering
\ifLuaTeX
  \usepackage{selnolig}  % disable illegal ligatures
\fi
\IfFileExists{bookmark.sty}{\usepackage{bookmark}}{\usepackage{hyperref}}
\IfFileExists{xurl.sty}{\usepackage{xurl}}{} % add URL line breaks if available
\urlstyle{same}
\hypersetup{
  pdftitle={sheet 2 solutions (coding part)},
  pdfauthor={carsten stahl},
  hidelinks,
  pdfcreator={LaTeX via pandoc}}

\title{sheet 2 solutions (coding part)}
\author{carsten stahl}
\date{2023-10-17}

\begin{document}
\maketitle

\hypertarget{loading-the-data}{%
\subsection{Loading the data}\label{loading-the-data}}

Before starting this exercise we are going to load the data and
requirements

\begin{Shaded}
\begin{Highlighting}[]
\FunctionTok{library}\NormalTok{(tidyverse)}
\end{Highlighting}
\end{Shaded}

\begin{verbatim}
## -- Attaching core tidyverse packages ------------------------ tidyverse 2.0.0 --
## v dplyr     1.1.2     v readr     2.1.4
## v forcats   1.0.0     v stringr   1.5.0
## v ggplot2   3.4.2     v tibble    3.2.1
## v lubridate 1.9.2     v tidyr     1.3.0
## v purrr     1.0.1     
## -- Conflicts ------------------------------------------ tidyverse_conflicts() --
## x dplyr::filter() masks stats::filter()
## x dplyr::lag()    masks stats::lag()
## i Use the conflicted package (<http://conflicted.r-lib.org/>) to force all conflicts to become errors
\end{verbatim}

\begin{Shaded}
\begin{Highlighting}[]
\FunctionTok{library}\NormalTok{(nlme)}
\end{Highlighting}
\end{Shaded}

\begin{verbatim}
## 
## Attache Paket: 'nlme'
## 
## Das folgende Objekt ist maskiert 'package:dplyr':
## 
##     collapse
\end{verbatim}

\begin{Shaded}
\begin{Highlighting}[]
\FunctionTok{load}\NormalTok{(}\StringTok{"./data/SimulatedTreatmentEffect.RData"}\NormalTok{)}
\end{Highlighting}
\end{Shaded}

\hypertarget{a}{%
\subsection{a)}\label{a}}

\begin{Shaded}
\begin{Highlighting}[]
\NormalTok{gnls\_treatment }\OtherTok{\textless{}{-}} \FunctionTok{gnls}\NormalTok{(resp }\SpecialCharTok{\textasciitilde{}}\NormalTok{ th1}\SpecialCharTok{+}\NormalTok{(th4}\SpecialCharTok{{-}}\NormalTok{th1)}\SpecialCharTok{/}\NormalTok{(}\DecValTok{1}\SpecialCharTok{+}\NormalTok{(}\FunctionTok{exp}\NormalTok{((conc}\SpecialCharTok{{-}}\NormalTok{th2)}\SpecialCharTok{*}\NormalTok{th3))),}
                       \AttributeTok{data =}\NormalTok{ conc.resp.df,}
                       \AttributeTok{params=}\FunctionTok{list}\NormalTok{(th1}\SpecialCharTok{+}\NormalTok{th2}\SpecialCharTok{+}\NormalTok{th3}\SpecialCharTok{+}\NormalTok{th4}\SpecialCharTok{\textasciitilde{}}\DecValTok{1}\NormalTok{),}
                       \AttributeTok{control=}\FunctionTok{gnlsControl}\NormalTok{(}\AttributeTok{nlsTol=}\FloatTok{0.1}\NormalTok{),}
                       \AttributeTok{start=}\FunctionTok{c}\NormalTok{(}\DecValTok{0}\NormalTok{, }\DecValTok{2}\NormalTok{, }\DecValTok{1}\NormalTok{, }\DecValTok{100}\NormalTok{))}

\FunctionTok{summary}\NormalTok{(gnls\_treatment)}
\end{Highlighting}
\end{Shaded}

\begin{verbatim}
## Generalized nonlinear least squares fit
##   Model: resp ~ th1 + (th4 - th1)/(1 + (exp((conc - th2) * th3))) 
##   Data: conc.resp.df 
##        AIC      BIC   logLik
##   313.2579 322.6139 -151.629
## 
## Coefficients:
##        Value Std.Error  t-value p-value
## th1  2.73936 2.8128950  0.97386  0.3355
## th2  1.82248 0.0866156 21.04101  0.0000
## th3  1.36932 0.1414898  9.67787  0.0000
## th4 98.13863 1.4919776 65.77755  0.0000
## 
##  Correlation: 
##     th1    th2    th3   
## th2 -0.643              
## th3  0.668 -0.299       
## th4 -0.228 -0.229 -0.478
## 
## Standardized residuals:
##        Min         Q1        Med         Q3        Max 
## -1.8609042 -0.6092146 -0.2138574  0.4650925  3.1081192 
## 
## Residual standard error: 5.950669 
## Degrees of freedom: 48 total; 44 residual
\end{verbatim}

\begin{itemize}
\tightlist
\item
  The slope is positive -\textgreater{} upward slope
\item
  The response ranges from 2 to almost 100
\item
  The alert-dosage (logscaled) is at around 2 (a little smaller than 1)
\end{itemize}

\hypertarget{b}{%
\subsection{b)}\label{b}}

\begin{Shaded}
\begin{Highlighting}[]
\CommentTok{\# creating dummy variables}
\NormalTok{conc.resp.df}\SpecialCharTok{$}\NormalTok{in\_T1 }\OtherTok{\textless{}{-}}\NormalTok{ conc.resp.df}\SpecialCharTok{$}\NormalTok{treat }\SpecialCharTok{==} \StringTok{"T1"}
\NormalTok{conc.resp.df}\SpecialCharTok{$}\NormalTok{in\_T2 }\OtherTok{\textless{}{-}}\NormalTok{ conc.resp.df}\SpecialCharTok{$}\NormalTok{treat }\SpecialCharTok{==} \StringTok{"T2"}

\CommentTok{\# applying model}
\NormalTok{gnls\_treatment\_with\_dummy }\OtherTok{\textless{}{-}} \FunctionTok{gnls}\NormalTok{(resp}\SpecialCharTok{\textasciitilde{}}\NormalTok{(th1}\SpecialCharTok{+}\NormalTok{(th4}\SpecialCharTok{{-}}\NormalTok{th1)}\SpecialCharTok{/}\NormalTok{(}\DecValTok{1}\SpecialCharTok{+}\NormalTok{(}\FunctionTok{exp}\NormalTok{((conc}\SpecialCharTok{{-}}\NormalTok{th2)}\SpecialCharTok{*}\NormalTok{th3))))}\SpecialCharTok{*}\NormalTok{in\_T1 }\SpecialCharTok{+}
\NormalTok{                                    (th1}\SpecialCharTok{+}\NormalTok{(th4}\SpecialCharTok{{-}}\NormalTok{th1)}\SpecialCharTok{/}\NormalTok{(}\DecValTok{1}\SpecialCharTok{+}\NormalTok{(}\FunctionTok{exp}\NormalTok{((conc}\SpecialCharTok{{-}}\NormalTok{th2)}\SpecialCharTok{*}\NormalTok{th3))))}\SpecialCharTok{*}\NormalTok{in\_T2,}
                       \AttributeTok{data =}\NormalTok{ conc.resp.df,}
                       \AttributeTok{params=}\FunctionTok{list}\NormalTok{(th1}\SpecialCharTok{+}\NormalTok{th2}\SpecialCharTok{+}\NormalTok{th3}\SpecialCharTok{+}\NormalTok{th4}\SpecialCharTok{\textasciitilde{}}\DecValTok{1}\NormalTok{),}
                       \AttributeTok{control=}\FunctionTok{gnlsControl}\NormalTok{(}\AttributeTok{nlsTol=}\FloatTok{0.1}\NormalTok{),}
                       \AttributeTok{start=}\FunctionTok{c}\NormalTok{(}\DecValTok{0}\NormalTok{, }\DecValTok{2}\NormalTok{, }\DecValTok{1}\NormalTok{, }\DecValTok{100}\NormalTok{))}

\FunctionTok{summary}\NormalTok{(gnls\_treatment\_with\_dummy)}
\end{Highlighting}
\end{Shaded}

\begin{verbatim}
## Generalized nonlinear least squares fit
##   Model: resp ~ (th1 + (th4 - th1)/(1 + (exp((conc - th2) * th3)))) *      in_T1 + (th1 + (th4 - th1)/(1 + (exp((conc - th2) * th3)))) *      in_T2 
##   Data: conc.resp.df 
##        AIC      BIC   logLik
##   313.2579 322.6139 -151.629
## 
## Coefficients:
##        Value Std.Error  t-value p-value
## th1  2.73936 2.8128950  0.97386  0.3355
## th2  1.82248 0.0866156 21.04101  0.0000
## th3  1.36932 0.1414898  9.67787  0.0000
## th4 98.13863 1.4919776 65.77755  0.0000
## 
##  Correlation: 
##     th1    th2    th3   
## th2 -0.643              
## th3  0.668 -0.299       
## th4 -0.228 -0.229 -0.478
## 
## Standardized residuals:
##        Min         Q1        Med         Q3        Max 
## -1.8609042 -0.6092146 -0.2138574  0.4650925  3.1081192 
## 
## Residual standard error: 5.950669 
## Degrees of freedom: 48 total; 44 residual
\end{verbatim}

Nothing really changed

\hypertarget{c}{%
\subsection{c)}\label{c}}

\begin{Shaded}
\begin{Highlighting}[]
\NormalTok{gnls\_treatment\_with\_dummy\_diff }\OtherTok{\textless{}{-}} \FunctionTok{gnls}\NormalTok{(resp}\SpecialCharTok{\textasciitilde{}}\NormalTok{(th1}\SpecialCharTok{+}\NormalTok{(th4}\SpecialCharTok{{-}}\NormalTok{th1)}\SpecialCharTok{/}\NormalTok{(}\DecValTok{1}\SpecialCharTok{+}\NormalTok{(}\FunctionTok{exp}\NormalTok{((conc}\SpecialCharTok{{-}}\NormalTok{th21)}\SpecialCharTok{*}\NormalTok{th31))))}\SpecialCharTok{*}\NormalTok{in\_T1 }\SpecialCharTok{+}
\NormalTok{                                    (th1}\SpecialCharTok{+}\NormalTok{(th4}\SpecialCharTok{{-}}\NormalTok{th1)}\SpecialCharTok{/}\NormalTok{(}\DecValTok{1}\SpecialCharTok{+}\NormalTok{(}\FunctionTok{exp}\NormalTok{((conc}\SpecialCharTok{{-}}\NormalTok{th22)}\SpecialCharTok{*}\NormalTok{th32))))}\SpecialCharTok{*}\NormalTok{in\_T2,}
                                 \AttributeTok{data =}\NormalTok{ conc.resp.df,}
                                 \AttributeTok{params=}\FunctionTok{list}\NormalTok{(th1}\SpecialCharTok{+}\NormalTok{th21}\SpecialCharTok{+}\NormalTok{th22}\SpecialCharTok{+}\NormalTok{th31}\SpecialCharTok{+}\NormalTok{th32}\SpecialCharTok{+}\NormalTok{th4}\SpecialCharTok{\textasciitilde{}}\DecValTok{1}\NormalTok{),}
                                 \AttributeTok{control=}\FunctionTok{gnlsControl}\NormalTok{(}\AttributeTok{nlsTol=}\FloatTok{0.1}\NormalTok{),}
                                 \AttributeTok{start=}\FunctionTok{c}\NormalTok{(}\DecValTok{0}\NormalTok{, }\DecValTok{2}\NormalTok{, }\DecValTok{2}\NormalTok{, }\DecValTok{1}\NormalTok{, }\DecValTok{1}\NormalTok{, }\DecValTok{100}\NormalTok{))}

\FunctionTok{summary}\NormalTok{(gnls\_treatment\_with\_dummy\_diff)}
\end{Highlighting}
\end{Shaded}

\begin{verbatim}
## Generalized nonlinear least squares fit
##   Model: resp ~ (th1 + (th4 - th1)/(1 + (exp((conc - th21) * th31)))) *      in_T1 + (th1 + (th4 - th1)/(1 + (exp((conc - th22) * th32)))) *      in_T2 
##   Data: conc.resp.df 
##        AIC      BIC    logLik
##   306.0199 319.1183 -146.0099
## 
## Coefficients:
##         Value Std.Error  t-value p-value
## th1   3.00020 2.5133967  1.19368  0.2393
## th21  1.65162 0.1010026 16.35222  0.0000
## th22  1.94849 0.0865419 22.51499  0.0000
## th31  1.23366 0.1339484  9.20998  0.0000
## th32  1.54790 0.1817802  8.51524  0.0000
## th4  98.34782 1.3582122 72.40977  0.0000
## 
##  Correlation: 
##      th1    th21   th22   th31   th32  
## th21 -0.546                            
## th22 -0.506  0.381                     
## th31  0.560 -0.213 -0.194              
## th32  0.533 -0.200 -0.172  0.377       
## th4  -0.222 -0.202 -0.195 -0.402 -0.387
## 
## Standardized residuals:
##        Min         Q1        Med         Q3        Max 
## -1.8025729 -0.5448639 -0.1250426  0.4162767  3.3504426 
## 
## Residual standard error: 5.417861 
## Degrees of freedom: 48 total; 42 residual
\end{verbatim}

Both alert-concentration and slope have changed after treatment. The
model is expected to have a better fit.

\hypertarget{d}{%
\subsection{d)}\label{d}}

\begin{Shaded}
\begin{Highlighting}[]
\FunctionTok{anova}\NormalTok{(gnls\_treatment\_with\_dummy\_diff, gnls\_treatment)}
\end{Highlighting}
\end{Shaded}

\begin{verbatim}
##                                Model df      AIC      BIC   logLik   Test
## gnls_treatment_with_dummy_diff     1  7 306.0199 319.1183 -146.010       
## gnls_treatment                     2  5 313.2579 322.6139 -151.629 1 vs 2
##                                 L.Ratio p-value
## gnls_treatment_with_dummy_diff                 
## gnls_treatment                 11.23801  0.0036
\end{verbatim}

Yes, the p value is suffiently low to say, that the models are
different.

\end{document}
